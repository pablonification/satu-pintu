\documentclass[11pt,a4paper]{article}

% Packages
\usepackage[utf8]{inputenc}
\usepackage[T1]{fontenc}
\usepackage[indonesian]{babel}
\usepackage{geometry}
\usepackage{graphicx}
\usepackage{xcolor}
\usepackage{hyperref}
\usepackage{fancyhdr}
\usepackage{titlesec}
\usepackage{enumitem}
\usepackage{tabularx}
\usepackage{booktabs}
\usepackage{longtable}
\usepackage{listings}
\usepackage{fancyvrb}
\usepackage{tcolorbox}
\usepackage{tikz}
\usepackage{pgfplots}
\usepackage{fontawesome5}
\usepackage{tocloft}

% Page geometry
\geometry{margin=2.5cm}

% Colors
\definecolor{primary}{RGB}{30,64,175}
\definecolor{secondary}{RGB}{59,130,246}
\definecolor{success}{RGB}{22,163,74}
\definecolor{warning}{RGB}{202,138,4}
\definecolor{danger}{RGB}{220,38,38}
\definecolor{codebg}{RGB}{248,250,252}
\definecolor{codeframe}{RGB}{226,232,240}

% Hyperref setup
\hypersetup{
    colorlinks=true,
    linkcolor=primary,
    filecolor=primary,
    urlcolor=secondary,
    pdftitle={SatuPintu - Dokumentasi Lengkap},
    pdfauthor={SatuPintu Team},
}

% Header and Footer
\pagestyle{fancy}
\fancyhf{}
\fancyhead[L]{\textcolor{primary}{\textbf{SatuPintu}}}
\fancyhead[R]{\textcolor{gray}{Dokumentasi Teknis}}
\fancyfoot[C]{\thepage}
\renewcommand{\headrulewidth}{0.4pt}
\renewcommand{\footrulewidth}{0.4pt}

% Section formatting
\titleformat{\section}
{\color{primary}\normalfont\Large\bfseries}
{\thesection}{1em}{}

\titleformat{\subsection}
{\color{secondary}\normalfont\large\bfseries}
{\thesubsection}{1em}{}

% Code listing style
\lstset{
    backgroundcolor=\color{codebg},
    basicstyle=\ttfamily\small,
    breaklines=true,
    frame=single,
    rulecolor=\color{codeframe},
    numbers=left,
    numberstyle=\tiny\color{gray},
    keywordstyle=\color{primary}\bfseries,
    commentstyle=\color{success},
    stringstyle=\color{warning},
    showstringspaces=false,
    tabsize=2,
}

% SQL language definition
\lstdefinelanguage{SQL}{
    keywords={CREATE, TABLE, IF, NOT, EXISTS, PRIMARY, KEY, DEFAULT, REFERENCES, ON, DELETE, CASCADE, SET, NULL, INSERT, INTO, VALUES, SELECT, FROM, WHERE, AND, OR, UPDATE, DROP, ALTER, INDEX, USING, GIN, ARRAY, CONSTRAINT, CHECK, IN, TRIGGER, FUNCTION, RETURNS, AS, BEGIN, END, LANGUAGE, BEFORE, AFTER, FOR, EACH, ROW, EXECUTE, NEW, OLD, RETURN},
    sensitive=false,
    morecomment=[l]{--},
    morestring=[b]',
}

% Tcolorbox styles
\tcbuselibrary{skins,breakable}

\newtcolorbox{infobox}{
    colback=blue!5,
    colframe=primary,
    title=Info,
    fonttitle=\bfseries,
    breakable
}

\newtcolorbox{warningbox}{
    colback=yellow!5,
    colframe=warning,
    title=Perhatian,
    fonttitle=\bfseries,
    breakable
}

\newtcolorbox{codebox}{
    colback=codebg,
    colframe=codeframe,
    breakable,
    left=0pt,
    right=0pt,
    top=0pt,
    bottom=0pt
}

% Document info
\title{
    \vspace{-2cm}
    {\Huge\textcolor{primary}{\textbf{SatuPintu}}}\\[0.5cm]
    {\Large\textcolor{secondary}{Satu Pintu untuk Semua Keluhan Kota}}\\[1cm]
    {\large Dokumentasi Teknis Lengkap}\\[0.5cm]
    {\normalsize Ekraf Tech Summit 2025 - Tech Innovation Challenge}
}
\author{SatuPintu Team}
\date{3 Desember 2025}

\begin{document}

% Title page
\maketitle
\thispagestyle{empty}

\vfill

\begin{center}
\begin{tikzpicture}
    \node[draw=primary, fill=primary!10, rounded corners=10pt, inner sep=15pt, text width=12cm, align=center] {
        \textbf{\large AI-Powered Centralized Call Center}\\[0.5em]
        Warga cukup menelepon satu nomor, AI memahami keluhan dan meneruskan ke dinas terkait dengan tracking otomatis.
    };
\end{tikzpicture}
\end{center}

\vfill

\newpage

% Table of Contents
\tableofcontents
\newpage

%===========================================
% PART 1: PRD
%===========================================
\part{Product Requirements Document}

\section{Executive Summary}

\begin{tabularx}{\textwidth}{lX}
\textbf{Nama Produk:} & SatuPintu \\
\textbf{Tagline:} & "Satu Nomor untuk Semua Kebutuhan Kota" \\
\textbf{Target Hackathon:} & Ekraf Tech Summit 2025 - Tech Innovation Challenge \\
\textbf{Kategori:} & Emergency Response and Smart City Solutions \\
\textbf{Demo Date:} & 17 Desember 2025 \\
\end{tabularx}

\subsection{Problem Statement}

Saat ini warga kota harus menghafal banyak nomor telepon untuk berbagai kebutuhan:

\begin{itemize}
    \item 112 - Darurat umum
    \item 110 - Polisi
    \item 119 - Ambulans
    \item 113 - Pemadam Kebakaran
    \item Dan puluhan nomor dinas lainnya...
\end{itemize}

\textbf{Pain Points:}
\begin{enumerate}
    \item Warga bingung harus menghubungi nomor mana
    \item Tidak ada mekanisme tracking status laporan
    \item Warga harus menelepon berulang kali untuk update
    \item Data laporan tersebar di berbagai instansi
    \item Tidak ada analytics untuk pengambilan keputusan
\end{enumerate}

\subsection{Solution}

SatuPintu adalah \textbf{AI-powered centralized call center} yang:
\begin{enumerate}
    \item Menyediakan satu nomor telepon untuk semua kebutuhan
    \item AI agent menerima dan memahami keluhan warga
    \item Otomatis mengkategorikan dan meneruskan ke dinas terkait
    \item Memberikan ticket ID untuk tracking
    \item Warga bisa cek status via web atau SMS
    \item Dashboard untuk setiap dinas mengelola laporan
\end{enumerate}

\section{Target Users}

\subsection{Primary Users}

\begin{tabularx}{\textwidth}{|l|X|X|}
\hline
\textbf{User Type} & \textbf{Deskripsi} & \textbf{Kebutuhan Utama} \\
\hline
Warga/Citizen & Masyarakat umum yang ingin melaporkan masalah & Mudah melapor, bisa tracking status \\
\hline
Operator Dinas & Staf dinas yang handle laporan & Menerima, update, dan close ticket \\
\hline
Admin Kota & Pejabat pemkot yang monitor & Overview analytics dan performa \\
\hline
\end{tabularx}

\subsection{User Personas}

\begin{tcolorbox}[colback=blue!5, colframe=primary, title=Persona 1: Pak Budi (55 tahun)]
\begin{itemize}[noitemsep]
    \item Warga biasa, tidak tech-savvy
    \item Ingin melaporkan lampu jalan mati
    \item Tidak tahu harus telepon kemana
    \item Butuh konfirmasi bahwa laporannya diproses
\end{itemize}
\end{tcolorbox}

\begin{tcolorbox}[colback=blue!5, colframe=primary, title=Persona 2: Ibu Sari (32 tahun)]
\begin{itemize}[noitemsep]
    \item Menyaksikan kecelakaan
    \item Panik, butuh bantuan cepat
    \item Ingin tahu apakah ambulans sudah dikirim
    \item Ingin update kondisi korban
\end{itemize}
\end{tcolorbox}

\section{Features \& Requirements}

\subsection{MVP Features (Must Have)}

\subsubsection{Voice AI Agent}
\begin{tabularx}{\textwidth}{|l|X|c|}
\hline
\textbf{Feature} & \textbf{Deskripsi} & \textbf{Priority} \\
\hline
Receive Call & Menerima telepon masuk via Twilio & P0 \\
\hline
Speech Recognition & Memahami bahasa Indonesia & P0 \\
\hline
Intent Classification & Kategorikan jenis laporan & P0 \\
\hline
Information Extraction & Ekstrak lokasi, detail, urgensi & P0 \\
\hline
Natural Response & Merespons dengan natural & P0 \\
\hline
Ticket Creation & Buat ticket otomatis & P0 \\
\hline
\end{tabularx}

\subsubsection{Ticket System}
\begin{tabularx}{\textwidth}{|l|X|c|}
\hline
\textbf{Feature} & \textbf{Deskripsi} & \textbf{Priority} \\
\hline
Generate Ticket ID & Format: SP-YYYYMMDD-XXXX & P0 \\
\hline
Store Ticket Data & Simpan ke database & P0 \\
\hline
Assign to Dinas & Auto-assign berdasarkan kategori & P0 \\
\hline
SMS Notification & Kirim ticket ID ke pelapor & P0 \\
\hline
Status Tracking & Pending → In Progress → Resolved & P0 \\
\hline
\end{tabularx}

\section{Kategori Laporan}

\begin{tabularx}{\textwidth}{|c|l|l|X|c|}
\hline
\textbf{Kode} & \textbf{Kategori} & \textbf{Dinas} & \textbf{Contoh} & \textbf{Urgensi} \\
\hline
DARURAT & Keadaan Darurat & Polisi/Damkar/Ambulans & Kecelakaan, kebakaran & CRITICAL \\
\hline
INFRA & Infrastruktur & Dinas PUPR & Jalan rusak, lampu mati & MEDIUM \\
\hline
KEBERSIHAN & Kebersihan & DLH & Sampah menumpuk & LOW \\
\hline
SOSIAL & Sosial & Dinsos & ODGJ, pengemis & MEDIUM \\
\hline
LAINNYA & Lain-lain & Admin Umum & Pertanyaan umum & LOW \\
\hline
\end{tabularx}

\subsection{Urgensi Level}
\begin{itemize}
    \item \textcolor{danger}{\textbf{CRITICAL}}: Response dalam 15 menit (darurat)
    \item \textcolor{warning}{\textbf{HIGH}}: Response dalam 1 jam
    \item \textcolor{secondary}{\textbf{MEDIUM}}: Response dalam 24 jam
    \item \textcolor{success}{\textbf{LOW}}: Response dalam 72 jam
\end{itemize}

\section{Tech Stack}

\begin{tabularx}{\textwidth}{|l|l|X|}
\hline
\textbf{Layer} & \textbf{Technology} & \textbf{Alasan} \\
\hline
Framework & Next.js 14 (App Router) & Fullstack, fast, modern \\
\hline
Database & Supabase (PostgreSQL) & Free tier, realtime, auth \\
\hline
Voice & Twilio Voice & Industry standard \\
\hline
AI/LLM & Gemini 2.0 Flash & Native audio, fast, cost-effective \\
\hline
TTS & Google Cloud TTS & Natural Indonesian voice \\
\hline
SMS & Twilio SMS & Reliable, included in trial \\
\hline
Hosting & Vercel & Zero config, free tier \\
\hline
UI & Tailwind + shadcn/ui & Rapid development \\
\hline
\end{tabularx}

\section{Success Metrics}

\subsection{Demo Metrics (untuk Hackathon)}
\begin{itemize}
    \item[$\square$] Bisa menerima telepon dan memahami intent
    \item[$\square$] Bisa membuat ticket dan kirim SMS
    \item[$\square$] Warga bisa cek status via web
    \item[$\square$] Dinas bisa update status ticket
    \item[$\square$] Demo berjalan lancar tanpa error
\end{itemize}

\section{Timeline}

\begin{tabularx}{\textwidth}{|c|c|X|}
\hline
\textbf{Day} & \textbf{Date} & \textbf{Deliverable} \\
\hline
1 & 3 Dec & PRD, User Flow, Wireframes, DB Schema \\
\hline
2 & 4 Dec & Project setup, DB setup, Basic UI \\
\hline
3 & 5 Dec & Voice AI integration \\
\hline
4 & 6 Dec & Ticket system + SMS \\
\hline
5 & 7 Dec & Citizen portal \\
\hline
6 & 8 Dec & Dinas dashboard \\
\hline
7 & 9 Dec & Integration testing \\
\hline
8 & 10 Dec & Bug fixes + Polish \\
\hline
9 & 11 Dec & Final submission \\
\hline
- & 16-17 Dec & Demo day! \\
\hline
\end{tabularx}

%===========================================
% PART 2: USER FLOW
%===========================================
\newpage
\part{User Flow Diagrams}

\section{Flow 1: Citizen Melaporkan via Telepon}

\begin{tcolorbox}[colback=blue!5, colframe=primary, title=Primary User Flow]
\begin{enumerate}
    \item \textbf{START:} Warga telepon ke SatuPintu
    \item \textbf{AI Greeting:} "Selamat datang di SatuPintu, layanan pengaduan terpadu Kota Bandung. Silakan sampaikan keluhan Anda."
    \item \textbf{Warga berbicara:} "Pak, ada kecelakaan di jalan Dago depan ITB, ada yang luka parah"
    \item \textbf{AI menganalisis:}
    \begin{itemize}
        \item Kategori: DARURAT
        \item Lokasi: Jalan Dago, depan ITB
        \item Urgensi: CRITICAL
        \item Dinas: Ambulans + Polisi
    \end{itemize}
    \item \textbf{AI konfirmasi:} "Baik, saya catat ada kecelakaan di Jalan Dago depan ITB dengan korban luka. Apakah benar?"
    \item \textbf{Warga:} "Ya, benar"
    \item \textbf{AI response:} "Laporan Anda sudah dicatat dengan nomor tiket SP-20251203-0001. Ambulans dan polisi sedang dikirim. Anda akan menerima SMS konfirmasi."
    \item \textbf{SMS dikirim} ke warga dengan ticket ID dan link tracking
    \item \textbf{END}
\end{enumerate}
\end{tcolorbox}

\section{Flow 2: Citizen Cek Status via Web}

\begin{enumerate}
    \item Warga buka website \texttt{satupintu.id}
    \item Masukkan nomor tiket: \texttt{SP-20251203-0001}
    \item Klik \textbf{CEK STATUS}
    \item Sistem menampilkan:
    \begin{itemize}
        \item Status: DALAM PROSES
        \item Kategori: Darurat - Kecelakaan
        \item Lokasi: Jl. Dago, depan ITB
        \item Timeline updates
    \end{itemize}
\end{enumerate}

\section{Flow 3: Citizen Cek Status via SMS}

\begin{enumerate}
    \item Warga menerima SMS konfirmasi setelah lapor
    \item Untuk cek status, reply: \texttt{CEK SP-20251203-0001}
    \item Sistem membalas dengan status terkini
\end{enumerate}

\section{Flow 4: Operator Dinas Mengelola Tiket}

\begin{enumerate}
    \item Operator login ke dashboard
    \item Melihat daftar tiket yang di-assign ke dinas-nya
    \item Klik tiket untuk melihat detail
    \item Update status (Pending → In Progress → Resolved)
    \item Tambah catatan
    \item SMS notifikasi dikirim ke warga
\end{enumerate}

\section{State Diagram: Ticket Status}

\begin{center}
\begin{tikzpicture}[
    node distance=3cm,
    state/.style={draw, rounded corners, minimum width=2.5cm, minimum height=1cm, fill=blue!10},
    arrow/.style={->, thick}
]
    \node[state, fill=yellow!20] (pending) {PENDING};
    \node[state, fill=blue!20, right of=pending] (progress) {IN\_PROGRESS};
    \node[state, fill=orange!20, below of=progress] (escalated) {ESCALATED};
    \node[state, fill=green!20, right of=progress] (resolved) {RESOLVED};
    \node[state, fill=gray!20, below of=pending] (cancelled) {CANCELLED};
    
    \draw[arrow] (pending) -- node[above] {diambil} (progress);
    \draw[arrow] (progress) -- node[above] {selesai} (resolved);
    \draw[arrow] (progress) -- node[right] {eskalasi} (escalated);
    \draw[arrow] (escalated) -- node[below right] {selesai} (resolved);
    \draw[arrow] (pending) -- node[left] {batal} (cancelled);
    \draw[arrow] (progress) -- node[below] {batal} (cancelled);
\end{tikzpicture}
\end{center}

%===========================================
% PART 3: WIREFRAMES
%===========================================
\newpage
\part{Screen Specifications}

\section{Screen Index}

\subsection{Public Screens (Citizen-facing)}
\begin{enumerate}
    \item Landing Page
    \item Ticket Status Page
    \item Error: Ticket Not Found
\end{enumerate}

\subsection{Dinas Dashboard Screens}
\begin{enumerate}
    \setcounter{enumi}{3}
    \item Dinas Login
    \item Dinas Dashboard
    \item Ticket Detail
\end{enumerate}

\section{Landing Page}

\textbf{URL:} \texttt{/} atau \texttt{satupintu.id}

\textbf{Purpose:} Entry point untuk warga mengecek status tiket

\textbf{Components:}
\begin{tabularx}{\textwidth}{|l|l|X|}
\hline
\textbf{Component} & \textbf{Type} & \textbf{Notes} \\
\hline
Logo & Image & SatuPintu logo \\
\hline
Tagline & Text & h1, prominent \\
\hline
Input Field & TextInput & Placeholder: "SP-..." \\
\hline
Submit Button & Primary Button & "CEK STATUS" \\
\hline
Help Section & Card & Phone number, 24/7 info \\
\hline
Category Icons & Icon + Text & Show supported categories \\
\hline
\end{tabularx}

\section{Ticket Status Page}

\textbf{URL:} \texttt{/track/[ticketId]}

\textbf{Components:}
\begin{itemize}
    \item Back Link
    \item Ticket ID (large, bold)
    \item Status Badge (color-coded)
    \item Detail Card (category, location, time, assignee)
    \item Timeline (vertical, chronological)
    \item Action Buttons
\end{itemize}

\textbf{Status Badge Colors:}
\begin{tabularx}{\textwidth}{|l|l|l|}
\hline
\textbf{Status} & \textbf{Color} & \textbf{Text} \\
\hline
PENDING & Red & Menunggu Penanganan \\
\hline
IN\_PROGRESS & Yellow & Dalam Proses \\
\hline
ESCALATED & Orange & Dieskalasi \\
\hline
RESOLVED & Green & Selesai \\
\hline
CANCELLED & Gray & Dibatalkan \\
\hline
\end{tabularx}

\section{Dinas Dashboard}

\textbf{URL:} \texttt{/dashboard}

\textbf{Components:}
\begin{itemize}
    \item Stats Cards (4 cards: Total, Pending, In Progress, Resolved)
    \item Filter Bar (Status, Urgensi, Tanggal, Search)
    \item Ticket List (paginated)
    \item Each ticket shows: ID, Category, Location, Status, Urgency, Time
\end{itemize}

\section{Design Tokens}

\subsection{Colors}
\begin{tabularx}{\textwidth}{|l|l|l|}
\hline
\textbf{Name} & \textbf{Hex} & \textbf{Usage} \\
\hline
Primary & \#1e40af & Main brand color \\
\hline
Primary Light & \#3b82f6 & Links, accents \\
\hline
Success & \#16a34a & Resolved status \\
\hline
Warning & \#ca8a04 & In Progress status \\
\hline
Error & \#dc2626 & Pending, errors \\
\hline
Background & \#f9fafb & Page background \\
\hline
\end{tabularx}

\subsection{Spacing}
\begin{itemize}
    \item xs: 4px
    \item sm: 8px
    \item md: 16px
    \item lg: 24px
    \item xl: 32px
\end{itemize}

%===========================================
% PART 4: API CONTRACT
%===========================================
\newpage
\part{API Contract}

\section{Overview}

Semua endpoint menggunakan REST API dengan JSON format.

\textbf{Base URL:}
\begin{itemize}
    \item Development: \texttt{http://localhost:3000/api}
    \item Production: \texttt{https://satupintu.vercel.app/api}
\end{itemize}

\section{Authentication Endpoints}

\subsection{POST /api/auth/login}

Login untuk operator dinas.

\textbf{Request:}
\begin{lstlisting}[language=json]
{
  "dinasId": "pupr",
  "password": "pupr2025"
}
\end{lstlisting}

\textbf{Response (200):}
\begin{lstlisting}[language=json]
{
  "success": true,
  "data": {
    "token": "eyJhbGciOiJIUzI1NiIsInR5cCI6IkpXVCJ9...",
    "dinas": {
      "id": "pupr",
      "name": "Dinas PUPR",
      "categories": ["INFRA"]
    }
  }
}
\end{lstlisting}

\section{Ticket Endpoints}

\subsection{GET /api/tickets}

Get list of tickets (untuk dashboard dinas).

\textbf{Query Parameters:}
\begin{tabularx}{\textwidth}{|l|l|l|X|}
\hline
\textbf{Param} & \textbf{Type} & \textbf{Required} & \textbf{Description} \\
\hline
status & string & No & Filter by status \\
\hline
urgency & string & No & Filter by urgency \\
\hline
category & string & No & Filter by category \\
\hline
page & number & No & Page number (default: 1) \\
\hline
limit & number & No & Items per page (default: 10) \\
\hline
\end{tabularx}

\subsection{GET /api/track/:ticketId}

Public endpoint untuk warga cek status (no auth required).

\textbf{Response (200):}
\begin{lstlisting}[language=json]
{
  "success": true,
  "data": {
    "id": "SP-20251203-0001",
    "category": "DARURAT",
    "status": "IN_PROGRESS",
    "statusText": "Dalam Proses",
    "location": "Jl. Dago, depan ITB",
    "timeline": [
      {"time": "2025-12-03T10:35:00Z", "message": "Korban dalam perjalanan ke RS"}
    ]
  }
}
\end{lstlisting}

\section{Voice Webhook}

\subsection{POST /api/voice/incoming}

Twilio webhook untuk incoming call.

\textbf{Response (TwiML):}
\begin{lstlisting}[language=xml]
<?xml version="1.0" encoding="UTF-8"?>
<Response>
  <Say voice="Google.id-ID-Standard-A">
    Selamat datang di SatuPintu...
  </Say>
  <Record maxLength="120" action="/api/voice/process" />
</Response>
\end{lstlisting}

\section{Data Types}

\subsection{Ticket Status}
\begin{lstlisting}[language=typescript]
type TicketStatus = 
  | "PENDING"      // Baru, belum ditangani
  | "IN_PROGRESS"  // Sedang ditangani
  | "ESCALATED"    // Dieskalasi
  | "RESOLVED"     // Selesai
  | "CANCELLED";   // Dibatalkan
\end{lstlisting}

\subsection{Ticket Category}
\begin{lstlisting}[language=typescript]
type TicketCategory = 
  | "DARURAT"      // Keadaan darurat
  | "INFRA"        // Infrastruktur
  | "KEBERSIHAN"   // Kebersihan
  | "SOSIAL"       // Masalah sosial
  | "LAINNYA";     // Lain-lain
\end{lstlisting}

\section{Error Responses}

Semua error mengikuti format:
\begin{lstlisting}[language=json]
{
  "success": false,
  "error": "Error message here",
  "code": "ERROR_CODE"
}
\end{lstlisting}

\begin{tabularx}{\textwidth}{|l|c|X|}
\hline
\textbf{Code} & \textbf{HTTP} & \textbf{Description} \\
\hline
UNAUTHORIZED & 401 & Token tidak valid \\
\hline
FORBIDDEN & 403 & Tidak punya akses \\
\hline
NOT\_FOUND & 404 & Resource tidak ditemukan \\
\hline
VALIDATION\_ERROR & 400 & Request body tidak valid \\
\hline
INTERNAL\_ERROR & 500 & Server error \\
\hline
\end{tabularx}

%===========================================
% PART 5: DATABASE SCHEMA
%===========================================
\newpage
\part{Database Schema}

\section{Overview}

Database menggunakan \textbf{Supabase PostgreSQL} dengan Row Level Security (RLS).

\section{Entity Relationship Diagram}

\begin{center}
\begin{tikzpicture}[
    entity/.style={draw, rectangle, minimum width=3cm, minimum height=2cm, fill=blue!10},
    relation/.style={draw, diamond, minimum width=1.5cm, fill=yellow!20}
]
    \node[entity] (dinas) at (0,0) {
        \textbf{dinas}\\
        \footnotesize id (PK)\\
        \footnotesize name\\
        \footnotesize categories[]
    };
    
    \node[entity] (tickets) at (5,0) {
        \textbf{tickets}\\
        \footnotesize id (PK)\\
        \footnotesize category\\
        \footnotesize status\\
        \footnotesize assigned\_dinas[]
    };
    
    \node[entity] (timeline) at (10,0) {
        \textbf{ticket\_timeline}\\
        \footnotesize id (PK)\\
        \footnotesize ticket\_id (FK)\\
        \footnotesize action\\
        \footnotesize message
    };
    
    \draw[->, thick] (dinas) -- node[above] {handles} (tickets);
    \draw[->, thick] (tickets) -- node[above] {has} (timeline);
\end{tikzpicture}
\end{center}

\section{Table: dinas}

Stores information about government agencies.

\begin{lstlisting}[language=SQL]
CREATE TABLE dinas (
  id VARCHAR(50) PRIMARY KEY,
  name VARCHAR(255) NOT NULL,
  password_hash VARCHAR(255) NOT NULL,
  categories TEXT[] NOT NULL DEFAULT '{}',
  phone VARCHAR(20),
  is_active BOOLEAN NOT NULL DEFAULT true,
  created_at TIMESTAMPTZ NOT NULL DEFAULT NOW(),
  updated_at TIMESTAMPTZ NOT NULL DEFAULT NOW()
);
\end{lstlisting}

\section{Table: tickets}

Main table for storing citizen reports.

\begin{lstlisting}[language=SQL]
CREATE TABLE tickets (
  id VARCHAR(20) PRIMARY KEY,
  category VARCHAR(20) NOT NULL,
  subcategory VARCHAR(100),
  location TEXT NOT NULL,
  description TEXT NOT NULL,
  reporter_phone VARCHAR(20) NOT NULL,
  status VARCHAR(20) NOT NULL DEFAULT 'PENDING',
  urgency VARCHAR(20) NOT NULL DEFAULT 'MEDIUM',
  assigned_dinas TEXT[] NOT NULL DEFAULT '{}',
  call_sid VARCHAR(100),
  transcription TEXT,
  created_at TIMESTAMPTZ NOT NULL DEFAULT NOW(),
  updated_at TIMESTAMPTZ NOT NULL DEFAULT NOW()
);
\end{lstlisting}

\section{Table: ticket\_timeline}

Stores timeline/history of ticket updates.

\begin{lstlisting}[language=SQL]
CREATE TABLE ticket_timeline (
  id UUID PRIMARY KEY DEFAULT gen_random_uuid(),
  ticket_id VARCHAR(20) NOT NULL REFERENCES tickets(id),
  action VARCHAR(50) NOT NULL,
  message TEXT NOT NULL,
  created_by VARCHAR(50) NOT NULL,
  is_public BOOLEAN NOT NULL DEFAULT true,
  metadata JSONB,
  created_at TIMESTAMPTZ NOT NULL DEFAULT NOW()
);
\end{lstlisting}

\section{Triggers \& Functions}

\subsection{Auto-update updated\_at}
\begin{lstlisting}[language=SQL]
CREATE OR REPLACE FUNCTION update_updated_at()
RETURNS TRIGGER AS $$
BEGIN
  NEW.updated_at = NOW();
  RETURN NEW;
END;
$$ LANGUAGE plpgsql;

CREATE TRIGGER trigger_tickets_updated_at
  BEFORE UPDATE ON tickets
  FOR EACH ROW
  EXECUTE FUNCTION update_updated_at();
\end{lstlisting}

%===========================================
% PART 6: PITCH DECK
%===========================================
\newpage
\part{Pitch Deck Outline}

\section{Overview}

Target durasi: \textbf{5-7 menit}\\
Total slides: \textbf{10-12 slides}

\section{Slide Structure}

\subsection{Slide 1: Title}
\begin{itemize}
    \item Logo SatuPintu
    \item Tagline: "AI-Powered Citizen Complaint Center"
    \item Ekraf Tech Summit 2025
    \item Tim
\end{itemize}

\subsection{Slide 2: Problem Statement}
\textbf{Warga Kebingungan Saat Ada Masalah}

Pain Points:
\begin{enumerate}
    \item Warga harus mengingat banyak nomor
    \item Saat panik, sulit menentukan harus telepon ke mana
    \item Laporan via telepon = tidak ada bukti \& tidak bisa dilacak
    \item Tidak tahu apakah laporan ditindaklanjuti
\end{enumerate}

\subsection{Slide 3: Solution Overview}
\textbf{SatuPintu: Satu Nomor, Semua Solusi}

Key Features:
\begin{enumerate}
    \item Satu nomor telepon untuk semua keluhan
    \item AI memahami dan mengkategorikan keluhan otomatis
    \item Tiket otomatis diteruskan ke dinas yang tepat
    \item Warga bisa melacak status via SMS atau web
\end{enumerate}

\subsection{Slide 4: How It Works}
Step-by-step demo flow dalam < 2 menit

\subsection{Slide 5: Live Demo}
\begin{warningbox}
Ini adalah slide terpenting! Siapkan backup video recording jika demo live gagal.
\end{warningbox}

\subsection{Slide 6: Technology Stack}
\begin{tabularx}{\textwidth}{|l|l|X|}
\hline
\textbf{Layer} & \textbf{Technology} & \textbf{Why} \\
\hline
Voice AI & Gemini 2.0 Flash & Native audio, fast \\
\hline
Voice Transport & Twilio & Reliable \\
\hline
Backend & Next.js 14 & Fullstack \\
\hline
Database & Supabase & PostgreSQL + Realtime \\
\hline
\end{tabularx}

\subsection{Slide 7: Key Differentiators}
\begin{tabularx}{\textwidth}{|X|X|}
\hline
\textbf{Traditional} & \textbf{SatuPintu} \\
\hline
Banyak nomor harus diingat & Satu nomor saja \\
\hline
Operator manusia (limited) & AI 24/7, unlimited \\
\hline
Tidak ada tracking & Real-time tracking \\
\hline
Data terpisah-pisah & Centralized dashboard \\
\hline
\end{tabularx}

\subsection{Slide 8: Market Opportunity}
\begin{itemize}
    \item 514 kota/kabupaten di Indonesia
    \item 275 juta penduduk
    \item 90\% keluhan tidak tertrack
\end{itemize}

\subsection{Slide 9: Roadmap}
\begin{itemize}
    \item Dec 2025: MVP + Hackathon
    \item Q1 2026: Pilot Bandung
    \item Q2 2026: Expand 3 cities
    \item Q3-Q4 2026: 10 cities target
\end{itemize}

\subsection{Slide 10: Team}
Profil tim dengan background

\subsection{Slide 11: Ask / CTA}
\begin{itemize}
    \item Vote for SatuPintu!
    \item Koneksi ke Pemkot Bandung
    \item Seed funding untuk scale
\end{itemize}

\subsection{Slide 12: Contact}
\begin{itemize}
    \item Website: satupintu.id
    \item Email: hello@satupintu.id
    \item QR code ke demo
\end{itemize}

\section{Timing Guide (7 min)}

\begin{tabularx}{\textwidth}{|l|c|}
\hline
\textbf{Section} & \textbf{Time} \\
\hline
Problem & 1 min \\
\hline
Solution & 1 min \\
\hline
Demo & 2 min \\
\hline
Tech \& Differentiators & 1 min \\
\hline
Market \& Roadmap & 1 min \\
\hline
Team \& Ask & 1 min \\
\hline
\end{tabularx}

\section{Key Messages}

\begin{enumerate}
    \item \textbf{"Satu nomor, semua solusi"} - repeat this tagline
    \item \textbf{"Warga bisa melacak keluhan mereka"} - key differentiator
    \item \textbf{"AI memahami bahasa Indonesia"} - tech innovation
    \item \textbf{"Proven model, localized"} - EffiGov validation
    \item \textbf{"Live dan bisa dicoba sekarang"} - working product
\end{enumerate}

%===========================================
% APPENDIX
%===========================================
\newpage
\appendix
\part*{Appendix}
\addcontentsline{toc}{part}{Appendix}

\section{Quick Start Guide}

\begin{enumerate}
    \item Install dependencies: \texttt{npm install}
    \item Setup environment variables (copy \texttt{.env.example})
    \item Setup Supabase database
    \item Setup Twilio webhooks
    \item Run: \texttt{npm run dev}
\end{enumerate}

\section{Demo Login Credentials}

\begin{tabularx}{\textwidth}{|l|l|}
\hline
\textbf{Dinas ID} & \textbf{Password} \\
\hline
admin & demo2025 \\
\hline
pupr & demo2025 \\
\hline
polisi & demo2025 \\
\hline
dlh & demo2025 \\
\hline
\end{tabularx}

\section{Environment Variables}

\begin{lstlisting}
# Supabase
NEXT_PUBLIC_SUPABASE_URL=your_url
NEXT_PUBLIC_SUPABASE_ANON_KEY=your_key
SUPABASE_SERVICE_ROLE_KEY=your_key

# Twilio
TWILIO_ACCOUNT_SID=your_sid
TWILIO_AUTH_TOKEN=your_token
TWILIO_PHONE_NUMBER=+1234567890

# Google AI
GOOGLE_AI_API_KEY=your_key

# App
NEXT_PUBLIC_APP_URL=http://localhost:3000
JWT_SECRET=your_secret
\end{lstlisting}

\vfill

\begin{center}
\textcolor{primary}{\rule{0.5\textwidth}{0.4pt}}\\[1em]
\textbf{\Large SatuPintu}\\
Satu Pintu untuk Semua Keluhan Kota\\[1em]
Ekraf Tech Summit 2025\\[2em]
\textit{Document Version 1.0 — December 2025}
\end{center}

\end{document}
